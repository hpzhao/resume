% !TEX TS-program = xelatex
% !TEX encoding = UTF-8 Unicode
% !Mode:: "TeX:UTF-8"

\documentclass{resume}
\usepackage{zh_CN-Adobefonts_external} % Simplified Chinese Support using external fonts (./fonts/zh_CN-Adobe/)
%\usepackage{zh_CN-Adobefonts_internal} % Simplified Chinese Support using system fonts
\usepackage{linespacing_fix} % disable extra space before next section
\usepackage{cite}

\begin{document}
\pagenumbering{gobble} % suppress displaying page number

\name{赵怀鹏}

\basicInfo{	
  \email{huaipengzhao@gmail.com} 
  \phone{(+86) 130-6987-3530} 
}
\section{\faGraduationCap\  教育背景}
\datedsubsection{\textbf{哈尔滨工业大学}}{2016  -- 至今}
\textit{在读硕士研究生}\ 社会计算与信息检索中心(SCIR),导师:车万翔, 预计 2018 年 7 月毕业
\datedsubsection{\textbf{哈尔滨工业大学}}{2012 -- 2016}
\textit{学士}\ 计算机科学与技术,学分绩排名前10\%

\section{\faUsers\ 项目/实习经历}


\datedsubsection{\textbf{CoNLL-17 Shared Task}}{2017年4月 -- 2017年5月}
\role{词性标注模块}{第五名}
主要研究多语言通用的词性标注方法,在64种语言整体评价获得\href{http://universaldependencies.org/conll17/results-upos.html}{第五名}。
\begin{itemize}
	\item 采用了两层RNN框架,第一层是character-level Bi-LSTM,用字符来获取词的表示,第二层是word-level Bi-LSTM, 该层的输入还包括词的布朗聚类以及预训练的word embedding。
\end{itemize}

\datedsubsection{\textbf{NLPCC-17 情感回复评测}}{2017年5月 -- 2017年6月}
\role{指定情感回复}{第二名}
给定一句话,生成指定情感的回复
\begin{itemize}
	\item 在传统的Seq2Seq模型上加入了3-hop Attention,在做Attention的时候加入了指定情感的embedding来控制生成的情感。另外利用了LTS来改善首字生成。
\end{itemize}

\datedsubsection{\textbf{中文顺滑}}{2016年9月 -- 2017年3月}
\role{语音后处理研究}{}
主要研究如何删除语音识别之后文本中冗余的词,使其通顺。
\begin{itemize}
  \item 主要尝试了Bi-LSTM-CRF和Pointer Network两种模型。后者的效果比前者的效果要好,达到了83\%左右的F值。

\end{itemize}

\datedsubsection{\textbf{讯飞研究院实习},北京}{2016年5月 -- 2016年9月}
\role{研究员}{负责语文主观题评测工作}
\begin{onehalfspacing}
对高考语文主观题进行机器打分,在定标实验下达到老师评分水平:
\begin{itemize}
  \item 采用了一些回归模型预测分数:SVR,Attention-Based-LSTM,得到的结果在0分一致率,1分一致率等指标上已经全面超过老师评分水平。
\end{itemize} 
\end{onehalfspacing}

\section{\faCogs\ IT 技能}
% increase linespacing [parsep=0.5ex]
\begin{itemize}[parsep=0.5ex]
  \item 编程语言: Python
  \item 研究领域: 自然语言处理,深度学习,机器学习
  
\end{itemize}

\section{\faHeartO\ 获奖情况}
\datedline{\textit{国家奖学金 }}{2013}

\section{\faInfo\ 其他}
% increase linespacing [parsep=0.5ex]
\begin{itemize}[parsep=0.5ex]
  \item 技术博客: \href{http://hpzhao.com}{http://hpzhao.com}
  \item 翻译书籍: \href{http://hpzhao.com/2016/07/10/%E3%80%90%E8%AF%91%E3%80%91%E7%A5%9E%E7%BB%8F%E7%BD%91%E7%BB%9C%E4%B8%8E%E6%B7%B1%E5%BA%A6%E5%AD%A6%E4%B9%A0/}{Deep Learning and Neural Networks}
  \item 外语: 英语六级
\end{itemize}

\end{document}
