% !TEX TS-program = xelatex
% !TEX encoding = UTF-8 Unicode
% !Mode:: "TeX:UTF-8"

\documentclass{resume}
\usepackage{zh_CN-Adobefonts_external} % Simplified Chinese Support using external fonts (./fonts/zh_CN-Adobe/)
%\usepackage{zh_CN-Adobefonts_internal} % Simplified Chinese Support using system fonts
\usepackage{linespacing_fix} % disable extra space before next section
\usepackage{cite}

\begin{document}
\pagenumbering{gobble} % suppress displaying page number

\name{徐伟}

% {E-mail}{mobilephone}{homepage}
% be careful of _ in emaill address
\contactInfo{readonlyfile@hotmail.com}{(+86) 152-1071-6440}{https://memeda.github.io}
% {E-mail}{mobilephone}
% keep the last empty braces!
%\contactInfo{xxx@yuanbin.me}{(+86) 131-221-87xxx}{}


求职基础信息

当前工作年限:~3年

当前工作地点:北京

期望入职时间:2020年7月+ 

期望工作地点:深圳

\section{\faUsers\ 工作经历}

\datedsubsection{\textbf{百度}}{2017年7月 -- 至今}
\expattr{高级工程师(T5)}{自然语言处理部}{北京}
\begin{onehalfspacing}
负责团队文本摘要子方向(共1$ \rightarrow $2人),持续迭代技术,有力支持业务
\begin{itemize}
  \item 独立设计、实现工业界领先的文本摘要系统;对内支持核心产品(Feed、小度等),对外赋能企业及个人(通过AI开放平台、智能创作平台等)
  \item 工作覆盖数据、评估、技术、落地、规划5个维度,关键积累100\%正式化,保证方向可持续发展
  \item 获部门奖项2次;指导、协同新人近1年;撰写专利4项
\end{itemize}
\end{onehalfspacing}

\section{\faUsers\ 项目经历}
\datedsubsection{}{}
\role{}{}

\section{\faGraduationCap\  教育背景}
\datedsubsection{\textbf{哈尔滨工业大学}, 哈尔滨,黑龙江}{2015年9月 -- 2017年7月}
\textit{硕士研究生}\ 计算机科学与技术, 社会计算与信息检索研究中心(导师:车万翔)
\datedsubsection{\textbf{哈尔滨工业大学}, 哈尔滨, 黑龙江}{2011年8月 -- 2015年7月}
\textit{学士}\ 计算机科学与技术 , 学分绩排名前10\%


\section{\faCogs\ IT 技能}
% increase linespacing [parsep=0.5ex]
\begin{itemize}[parsep=0.5ex]
  \item 编程语言: 偏好C++,熟练使用Python,有C,Shell,JS ,Java基础。
  \item 工程经验: 熟悉基础Linux脚本和WEB(前端)编写,做过一点QT、MFC、Java应用。
  \item 研究方向: 文本分类、序列标注(结构化预测),深度学习。
\end{itemize}

\section{\faHeartO\ 获奖情况}
\datedline{国家奖学金}{2014年10月}

\section{\faInfo\ 其他}
% increase linespacing [parsep=0.5ex]
\begin{itemize}[parsep=0.5ex]
  \item GitHub: https://github.com/memeda
  \item 外语: 英语6级
\end{itemize}

% Reference Test
%\datedsubsection{\textbf{Paper Title\cite{zaharia2012resilient}}}{May. 2015}
%An xxx optimized for xxx\cite{verma2015large}
%\begin{itemize}
%  \item main contribution
%\end{itemize}

%% Reference
%\newpage
%\bibliographystyle{IEEETran}
%\bibliography{mycite}
\end{document}
