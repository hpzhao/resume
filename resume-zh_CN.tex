% !TEX TS-program = xelatex
% !TEX encoding = UTF-8 Unicode
% !Mode:: "TeX:UTF-8"

\documentclass{resume}
\usepackage{zh_CN-Adobefonts_external} % Simplified Chinese Support using external fonts (./fonts/zh_CN-Adobe/)
%\usepackage{zh_CN-Adobefonts_internal} % Simplified Chinese Support using system fonts
\usepackage{linespacing_fix} % disable extra space before next section
\usepackage{cite}

\begin{document}
\pagenumbering{gobble} % suppress displaying page number

\name{徐伟}

% {E-mail}{mobilephone}{homepage}
% be careful of _ in emaill address
\contactInfo{readonlyfile@hotmail.com}{(+86) 180-0451-5446}{https://memeda.github.io}
% {E-mail}{mobilephone}
% keep the last empty braces!
%\contactInfo{xxx@yuanbin.me}{(+86) 131-221-87xxx}{}

\section{\faGraduationCap\  教育背景}
\datedsubsection{\textbf{哈尔滨工业大学}, 哈尔滨,黑龙江}{2015年9月 -- 至今}
\textit{在读硕士研究生}\ 计算机科学与技术, 预计 2017 年 7 月毕业
\datedsubsection{\textbf{哈尔滨工业大学}, 哈尔滨, 黑龙江}{2011年8月 -- 2015年7月}
\textit{学士}\ 计算机科学与技术 , 学分绩排名前10\%

\section{\faUsers\ 项目/实习经历}

\datedsubsection{\textbf{基于深度学习的中文分词、词性标注和命名实体识别}}{2016年3月 -- 至今}
\role{C++}{主要项目}
\begin{onehalfspacing}
https://github.com/memeda/sequence-labeling-by-nn

以构建准确率高、速度快的工业级序列标注系统为目标。
\begin{itemize}
  \item 融合深度表示学习(MLP、LSTM)与全局结构预测方法(CRF)。
  \item 利用C++语言的封装与抽象能力,编写完成不同输入、不同表示学习、不同输出预测的模型。
  \item 在与LTP(www.ltp-cloud.com)相同的评测数据集上完成大量实验,目前在词性标注和命名实体识别任务上相比基于传统结构化感知器的LTP 取得了更好的分数。
\end{itemize}
\end{onehalfspacing}

\datedsubsection{\textbf{面向不均匀类别的文本分类系统设计与实现}}{2015年3月 -- 2015年6月}
\role{Python}{本科毕业设计}
\begin{onehalfspacing}
https://github.com/memeda/GriduationDesignCodeForTextClassification

我们构建了两个文本分类系统,分别基于提升树(侧重鲁棒性)和支持向量机(倾向高准确率)。
\begin{itemize}
  \item 在中英文均衡和不均衡数据集上,通过组合不同特征、分类器做了大量的对比实验。
  \item 使用sampling、SMOTE、调整分类器权值的方法提升在不均衡数据集上的文本分类效果。
\end{itemize}
\end{onehalfspacing}

\datedsubsection{\textbf{百度实习},北京}{2014 年6月 -- 2014年9月}
\role{C/C++ , Python , shell , MapReduce}{自然语言处理部门}
独立完成知识库抽取系统在Hadoop集群上的部署。
\begin{itemize}
  \item 提升脚本处理性能;验证核心程序正确性。
  \item 在5个集群上同时部署系统,持续处理20K个百灵库。
  \item 历时一个月,抽取200+TB知识。
\end{itemize}

% Reference Test
%\datedsubsection{\textbf{Paper Title\cite{zaharia2012resilient}}}{May. 2015}
%An xxx optimized for xxx\cite{verma2015large}
%\begin{itemize}
%  \item main contribution
%\end{itemize}

\section{\faCogs\ IT 技能}
% increase linespacing [parsep=0.5ex]
\begin{itemize}[parsep=0.5ex]
  \item 编程语言: 偏好C++,熟练使用Python,有C,Shell,JS ,Java基础。
  \item 工程经验: 熟悉基础Linux脚本和WEB(前端)编写,做过一点QT、MFC、Java应用。
  \item 研究方向:分本分类、序列标注(结构化预测),深度学习。
\end{itemize}

\section{\faHeartO\ 获奖情况}
\datedline{国家奖学金}{2014年10月}

\section{\faInfo\ 其他}
% increase linespacing [parsep=0.5ex]
\begin{itemize}[parsep=0.5ex]
  \item GitHub: https://github.com/memeda
  \item 外语: 英语6级
\end{itemize}

%% Reference
%\newpage
%\bibliographystyle{IEEETran}
%\bibliography{mycite}
\end{document}
