% !TEX TS-program = xelatex
% !TEX encoding = UTF-8 Unicode
% !Mode:: "TeX:UTF-8"

\documentclass{resume}
\usepackage{zh_CN-Adobefonts_external} % Simplified Chinese Support using external fonts (./fonts/zh_CN-Adobe/)
%\usepackage{zh_CN-Adobefonts_internal} % Simplified Chinese Support using system fonts
\usepackage{linespacing_fix} % disable extra space before next section
\usepackage{cite}

\begin{document}
\pagenumbering{gobble} % suppress displaying page number

\name{赵怀鹏}

\basicInfo{	
  \email{huaipengzhao@gmail.com} 
  \phone{(+86) 130-6987-3530} 
}
\section{\faGraduationCap\  教育背景}
\datedsubsection{\textbf{哈尔滨工业大学}}{2016 -- 至今}
\textit{在读硕士研究生}\ 社会计算与信息检索中心(SCIR),导师:车万翔, 预计 2018 年 6 月毕业
\datedsubsection{\textbf{哈尔滨工业大学}}{2012 -- 2016}
\textit{学士}\ 计算机科学与技术,学分绩排名前10\%

\section{\faUsers\ 项目/实习经历}
\datedsubsection{\textbf{中文顺滑}}{2016年9月 -- 至今}
\role{语音后处理研究}{主要项目}
主要研究中文语音识别之后文本不规范的问题。主要尝试的方法有下面几种:
\begin{itemize}
  \item 基于序列标注的模型,这里尝试了CRF,BI-LSTM,BI-LSTM-CRF三种模型。实验结果来看后两种模型效果相似,但明显好于CRF,达到了83\%左右的F值。
  \item 基于生成的Pointer Network模型,这是基于seq2seq的改版,效果要略好于BI-LSTM-CRF。
  \item 联合模型,将LSTM的结果作为CRF的特征,做了Stacking操作,但是效果来看并没有提升。
  \item 将标点和顺滑联合做Multi-Task,但是结果也并没有显著提升,这主要由于语料问题导致有瓶颈。
\end{itemize}


\datedsubsection{\textbf{讯飞研究院实习},北京}{2016年5月 -- 2016年9月}
\role{研究员}{负责语文主观题评测工作}
\begin{onehalfspacing}
对高考语文主观题进行机器打分,在定标情况下达到人类老师评分水平:
\begin{itemize}
  \item 抽取了一些相似度离散特征,例如基于embedding的余弦相似度,编辑距离等。
  \item 采用了一些回归模型预测分数:SVR,Attention-Based-LSTM,并且也做了联合Stacking,从结果来看,联合模型得到的结果在0分一致率,1分一致率等指标上已经全面超过人类老师评分水平。
\end{itemize} 
\end{onehalfspacing}

\datedsubsection{\textbf{语文高考题干分析},863项目}{2016年3月 -- 2016年7月}
\role{本科毕设}{哈工大百优论文候选}
\begin{onehalfspacing}
	对高考语文题干进行分析,用来抽取答题引擎需要的信息。
	\begin{itemize}
		\item 抽取答案个数,出发词,约束条件等语义槽。
		\item 采用了SVM和CRF模型,实验结果证明该问题更适合序列标注模型。
	\end{itemize} 
\end{onehalfspacing}

\section{\faCogs\ IT 技能}
% increase linespacing [parsep=0.5ex]
\begin{itemize}[parsep=0.5ex]
  \item 编程语言: 常用语言:Python,有一定C++基础
  \item 研究领域: 文本相似度计算,中文顺滑,深度学习
  
\end{itemize}

\section{\faHeartO\ 获奖情况}
\datedline{\textit{国家奖学金 }}{2013}

\section{\faInfo\ 其他}
% increase linespacing [parsep=0.5ex]
\begin{itemize}[parsep=0.5ex]
  \item 技术博客: \href{http://hpzhao.com}{http://hpzhao.com}
  \item 翻译书籍: \href{http://hpzhao.com/2016/07/10/%E3%80%90%E8%AF%91%E3%80%91%E7%A5%9E%E7%BB%8F%E7%BD%91%E7%BB%9C%E4%B8%8E%E6%B7%B1%E5%BA%A6%E5%AD%A6%E4%B9%A0/}{Deep Learning and Neural Networks}
  \item 外语: 英语六级
\end{itemize}

\end{document}
